

\begin{problem}
Let $n = p_1p_2...p_k$ where $p_1,p_2,...,p_k$ are different primes.	
Prove that $n$ has exactly $2^k$ different divisors. Give a complete argument.

For example, if $n =105$ then $n = 3\cdot 5 \cdot 7$, so $k=3$ and
thus $n$ has $2^3 = 8$ divisors. These divisors are $1,3,5,7,15,21,35,105$.

\emph{Hint:} You can reduce the problem to counting other objects, that we already
know how to count. Alternatively, this can be proved by induction on $k$.
\end{problem}

\begin{solution}
\\
Using proof by induction, we can see why this theorem holds true.
\\
\\
\emph{Base case:} $n=2$
In this case $n=p_1$, so k=1. $2^k = 2^1 = 2$ We see that there are 2 unique prime
divisors; 1 and 2.
\\
\\
\emph{Assumption: } Let $n = p_1 \cdot p_2 \cdot ... \cdot p_k$ where $p_1,p_2,...,p_k$ are different primes.
We assume $n$ has exactly $2^k$ unique divisors.
\\
\\
\emph{Inductive step: } Let $m = p_1 \cdot p_2 \cdot ... \cdot p_k \cdot p_{(k+1)}$
After substitution, $m = n \cdot p_{(k+1)}$
\\
\\
If $p_1 | n$ and $p_2 | n$, then $(p_1 \cdot p_2) | n$. To find every divisor of $n$, each unique divisor
must be multiplied with each other. When $m = n \cdot p_{(k+1)}$, $p_{(k+1)}$ must be multiplied by every
divisor of $n$ to create a new unique set for $m$.
\\
\\
This only holds true for UNIQUE divisors, because non-unique numbers create overlaps in the final divisor set.
\\
\\
\emph{Example:} 
\[n=15 \]
\[ 15 = 3 \cdot 5 \]

There are 2 unique, prime divisors, so $2^k = 2^2 = 4$

The 4 divisors are $1, 3, 5, 15$

\[m=30 \]
\[ 30 = 3 \cdot 5 \cdot 2 \]

$p_{(k+1)}$ is $2$. There are now 3 unique, prime divisors, so $2^k = 2^3 = 8$

The 8 divisors are $1, 2, 3, 6, 5, 10, 15, 30$.

Each of the elements of the original 4 divisor set multiplied by $p_{(k+1)}$ yields the new 8 divisor set.

\[m=45 \]
\[45 = 3 \cdot 5 \cdot 3 \]

In this case, $p_{(k+1)}$ is $3$.  $2^k = 2^3 = 8$, but since $3$ is not unique, this does not apply.
This is proven by the factorization of $45$, which is $1, 3, 5, 9, 15, 45$.

$6 \neq 8$
\end{solution}
%%%%%%%%%%%%%%%%%%%%%%%%%%%%
