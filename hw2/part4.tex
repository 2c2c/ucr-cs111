
\begin{problem}
(a) Compute $8^{-1}\pmod{19}$ by enumerating multiples of the number and the modulus.
Show your work.

\smallskip\noindent
(b) Compute $8^{-1}\pmod{19}$ using Fermat's theorem. Show your work.

\smallskip\noindent
(c) Compute $20^{-1}\pmod{31}$ by enumerating multiples of the number and the modulus.
Show your work.

\smallskip\noindent
(d) Compute $20^{-1}\pmod{31}$ using Fermat's theorem. Show your work.

\smallskip\noindent
(e) Find an integer $x$, $0\le x \le 36$, that satisfies
$17x = 8 \pmod{37}$. Show your work.
\end{problem}

\begin{solution}
(a)

\begin{align*}
8^{-1} \pmod{19} &\equiv 1 \pmod{19}\\
8a &= 19b + 1\\
8a + 19b &= 1\\
\end{align*}
\smallskip\noindent

So we must find a and b that makes this equation hold:

$a = {8, 16, 24, 32, 40, 48, 56}$

$b = {19, 38, 57}$

So this holds for $a=-7, b=3$

So, 
\begin{align*}
8 \cdot -7 &= 19 \cdot b + 1\\
8 \cdot -7 &\equiv 1 \pmod{19}\\
8^{-1} &\equiv -7 \pmod{19}\\
&\equiv 12 \pmod{19}\\
&\equiv 12
\end{align*}

\pagebreak
(b)

\begin{align*}
1 \pmod{19} &\equiv 8^{18} \pmod{19}&& \text{By FLT}\\
8^{-1} \pmod{19} &\equiv 8^{-1} \cdot 8^{18} \pmod{19} && \text{multiply $8^{-1}$ to both sides}\\
\\
8^{17} \pmod{19} &\equiv 8^{16} \cdot 8 \pmod{19} \\
&\equiv 8^{16} \cdot 8 \pmod{19} \\
&\equiv 64^{8} \cdot 8 \pmod{19} \\
&\equiv 7^{8} \cdot 8 \pmod{19} \\
&\equiv 49^{4} \cdot 8 \pmod{19} \\
&\equiv 11^{4} \cdot 8 \pmod{19} \\
&\equiv 121^{2} \cdot 8 \pmod{19} \\
&\equiv 7^{2} \cdot 8 \pmod{19} \\
&\equiv 49 \cdot 8 \pmod{19} \\
&\equiv 11 \cdot 8 \pmod{19} \\
&\equiv 88 \pmod{19} \\
&\equiv 12 \pmod{19} \\
&\equiv 12 
\end{align*}

(c)

We must find
\begin{align*}
20a + 31b &= 1\\
\end{align*}

For some a and b.

$a = {20, 40, 60, 80, 100, 120, 140, 160, 180, 200, 220, 240, 260, 280}$

$b = {31, 62, 93, 124, 155, 186, 217, 248, 279}$

So this holds for $a=-7, b=3$

So, 

\begin{align*}
20 \cdot 14 &= 9 \cdot b + 1 \\
20 \cdot 14 &\equiv 1 \pmod{31} \\
20^{-1} &\equiv 14 \pmod{31} \\
&\equiv 14 
\end{align*}

\pagebreak
(d)

\begin{align*}
1 \pmod {31} &\equiv 20^{30} \pmod{31} \\ 
20^{-1} \pmod{31} &= 20^{-1} \cdot 20^{30} \pmod{31} \\
\\
20^{29} \pmod{31} &\equiv 20^{28} \cdot 20 \pmod{31} \\
&\equiv 400^{14} \cdot 20 \pmod{31} \\
&\equiv 28^{14} \cdot 20 \pmod{31} \\
&\equiv 784^{7} \cdot 20 \pmod{31} \\
&\equiv 9^{7} \cdot 20 \pmod{31} \\
&\equiv 9^{6} \cdot 180 \pmod{31} \\
&\equiv 81^{3} \cdot 180 \pmod{31} \\
&\equiv 19^{3} \cdot 180 \pmod{31} \\
&\equiv 19^{2} \cdot 19 \cdot 180 \pmod{31} \\
&\equiv 361 \cdot 19 \cdot 180 \pmod{31} \\
&\equiv 20 \cdot 19 \cdot 180 \pmod{31} \\
&\equiv 380 \cdot 180 \pmod{31} \\
&\equiv 8 \cdot 25 \pmod{31} \\
&\equiv 200 \pmod{31} \\
&\equiv 14 \pmod{31} \\
&\equiv 14
\end{align*}

\pagebreak
(e)

\begin{align*}
17x &\equiv 8 \pmod{37} \\
x &\equiv 8 \cdot 17^{-1} \pmod{37} \\
\\
17^{-1} \pmod{37} &\equiv 17^{35} \pmod{37} \\
&\equiv 17^{34} \cdot 17 \pmod{37} \\
&\equiv 289^{17} \cdot 17 \pmod{37} \\
&\equiv 30^{17} \cdot 17 \pmod{37} \\
&\equiv 30^{16} \cdot 30 \cdot 17 \pmod{37} \\
&\equiv 900^{8} \cdot 30 \cdot 17 \pmod{37} \\
&\equiv 12^{8} \cdot 30 \cdot 17 \pmod{37} \\
&\equiv 144^{4} \cdot 30 \cdot 17 \pmod{37} \\
&\equiv 33^{4} \cdot 30 \cdot 17 \pmod{37} \\
&\equiv 1089^{2} \cdot 30 \cdot 17 \pmod{37} \\
&\equiv 16^{2} \cdot 30 \cdot 17 \pmod{37} \\
&\equiv 256 \cdot 30 \cdot 17 \pmod{37} \\
&\equiv 34 \cdot 30 \cdot 17 \pmod{37} \\
&\equiv 510 \cdot 17 \pmod{37} \\
&\equiv 29 \cdot 17 \pmod{37} \\
&\equiv 24 \pmod{37} \\
\\
x &\equiv 8 \cdot 24 \pmod{37} \\
&\equiv 192 \pmod{37} \\
&\equiv 7 \pmod{37} \\
&\equiv 7
\end{align*}

\end{solution}
%%%%%%%%%%%%%%%%%%%%%%%%%%%%
