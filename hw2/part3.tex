
\begin{problem}
Alice's RSA public key is $P = (e,n) = (11,65)$.
Bob sends Alice the message by encoding it as follows.
First he assigns numbers to characters:
A is 2, B is 3, ..., Z is 27, and blank is 28. Then he
uses RSA to encode each number separately. 

Bob's encoded message is:

\begin{verbatim}
     31      29      11       7      60      30
     28      28      11      24      11      20
     49      11       7      22      11      31
     19      11      11      20       7      15
     31       3      23      30      60      30
     31      26       7      33      20      60
      7      57      11      20      30       3
     15       7      30      15       7      31
     29      33      31       7      57      11
     20      30       3      15       7      29
     33      15       7      30      31      15
      7      52      30      14      30      31
     15
\end{verbatim}

Decode Bob's message.
Notice that you don't have Bob's secrete key, so you
need to ``break" RSA to decrypt his message.

For the solution, give Bob's message in plaintext. (Also, who said it?)
You also need to describe step by step how you arrived at the solution. 

\emph{Suggestion:} this can be solved by hand, but it could get tedious. It may
be faster to write a short program.
\end{problem}

\begin{solution}

$P = (e,n) = (11,65)$

Since $p * q = n$, we know possible candidates are $1$ and $65$, or $5$ and $13$.
The first combination results in; 

\[ \varphi(n) = 0 \cdot 64 = 0 \]

This can't be correct since you cannot $mod 0$. So we know that;

\[ \varphi(n) = 4 \cdot 12 = 48 \]

We solve for the secret key using;

\[ d = 11^{-1}(mod \varphi(n)) \]
\[ d = 11^{-1}(mod 48) \]
\[ d = 35 \]

With $d$ and $n$, we can decrypt the message using;

\[ M = C^{35}(mod 65) \]
Where $M$ is the original message and $C$ is the encrypted version.
\end{solution}
%%%%%%%%%%%%%%%%%%%%%%%%%%%%
