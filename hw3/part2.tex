
\newcommand{\ttA}{\texttt{A}}
\newcommand{\ttB}{\texttt{B}}
\newcommand{\ttC}{\texttt{C}}
\newcommand{\ttE}{\texttt{E}}
\newcommand{\ttF}{\texttt{F}}

\begin{problem}
We form strings by concatenating strings $\ttA$, $\ttB\ttB$, $\ttB\ttC$, $\ttC\ttB$, $\ttC\ttC$,
$\ttE\ttE\ttF$ and $\ttF\ttE\ttE$. Let $S_n$ be the number of strings of length $n$ that can
be formed in this way. For example, for $n=3$, we can form the following strings:
%
\begin{align*}
\ttA\ttA\ttA
,
\ttA\ttB\ttB  , \ttA\ttB\ttC  , \ttA\ttC\ttB  , \ttA\ttC\ttC
,
\ttB\ttB\ttA  , \ttB\ttC\ttA  , \ttC\ttB\ttA  , \ttC\ttC\ttA 
,
\ttE\ttE\ttF , \ttF\ttE\ttE
\end{align*}
%
and thus $S_3 = 11$. (Note that $S_0 = 1$, because the
empty string satisfies the condition.)

\smallskip
\noindent (a) Derive a recurrence relation for
        the numbers $S_n$. Justify it.

\smallskip
\noindent (b) Find the formula for the numbers $S_n$
                by solving this recurrence.
                Show your work.
\end{problem}

\begin{solution}

To set up the recurrence relation, let's look at how strings can end.

- If $n$ string ends with "A", you have $(n-1)$ characters left to fill. Theres one root string ending in "A".
\newline

- If $n$ string ends with "B", you have $(n-2)$ characters left to fill. Theres two root strings ending in "B".
\newline

- If $n$ string ends with "C", you have $(n-2)$ characters left to fill. Theres two root strings ending in "C".
\newline

- If $n$ string ends with "E", you have $(n-3)$ characters left to fill. Theres one root string ending in "E".
\newline

- If $n$ string ends with "F", you have $(n-3)$ characters left to fill. Theres one root string ending in "F".
\newline

Using the logic above, we can create the relation:

\[ P_{n} = P_{n-3} + P_{n-3} + 2P_{n-2} + 2P_{n-2} + P_{n-1} \]

\[ P_{n} = 2P_{n-3} + 4P_{n-2} + P_{n-1} \]

Writing each solution out by hand, we solve the base cases.

\[ P_{0} = 1 \]
\[ P_{1} = 1 \]
\[ P_{2} = 5 \]
\[ P_{3} = 11 \]

Characteristic equation:

\[ P_{n} = x^{n} \]
\[ x^{n} = 2x^{n-3} + 4x^{n-2} + x^{n-1} \]
\[ x^{3} - x^{2} - 4x - 2 = 0 \]
\[ x = -1, 1 \pm \sqrt{3} \]

General equation:

\[ P_{n} = \alpha_{1}(-1)^{n} + \alpha_{2}(1-\sqrt{3})^{n} + \alpha_{3}(1 + \sqrt{3})^{n} \]

Using base cases, we construct the following system of equations:

\[ P_{0} = \alpha_{1} + \alpha_{2} + \alpha_{3} = 1 \]
\[ P_{1} = \alpha_{1}(-1) + \alpha_{2}(1-\sqrt{3}) + \alpha_{3}(1+\sqrt{3}) = 1 \]
\[ P_{5} = \alpha_{1}(-1)^{2} + \alpha_{2}(1-\sqrt{3})^{2} + \alpha_{3}(1+\sqrt{3})^{2} = 5 \]

Solving gives us:

\[ \alpha_{1} = 1 \]
\[ \alpha_{2} = \frac{-1}{\sqrt{3}} \]
\[ \alpha_{3} = \frac{1}{\sqrt{3}} \]

Plugging the constants gives us our final equation:

\[ P_{n} = (-1)^{n} - \frac{1}{\sqrt{3}}(1-\sqrt{3})^{n} + \frac{1}{\sqrt{3}}(1 + \sqrt{3})^{n} \]


\end{solution}
%%%%%%%%%%%%%%%%%%%%%%%%%%%%
