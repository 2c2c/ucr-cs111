
\newcommand{\ttA}{\texttt{A}}
\newcommand{\ttB}{\texttt{B}}
\newcommand{\ttC}{\texttt{C}}
\newcommand{\ttE}{\texttt{E}}
\newcommand{\ttF}{\texttt{F}}

\begin{problem}
We form strings by concatenating strings $\ttA$, $\ttB\ttB$, $\ttB\ttC$, $\ttC\ttB$, $\ttC\ttC$,
$\ttE\ttE\ttF$ and $\ttF\ttE\ttE$. Let $S_n$ be the number of strings of length $n$ that can
be formed in this way. For example, for $n=3$, we can form the following strings:
%
\begin{align*}
\ttA\ttA\ttA
,
\ttA\ttB\ttB  , \ttA\ttB\ttC  , \ttA\ttC\ttB  , \ttA\ttC\ttC
,
\ttB\ttB\ttA  , \ttB\ttC\ttA  , \ttC\ttB\ttA  , \ttC\ttC\ttA 
,
\ttE\ttE\ttF , \ttF\ttE\ttE
\end{align*}
%
and thus $S_3 = 11$. (Note that $S_0 = 1$, because the
empty string satisfies the condition.)

\smallskip
\noindent (a) Derive a recurrence relation for
        the numbers $S_n$. Justify it.

\smallskip
\noindent (b) Find the formula for the numbers $S_n$
                by solving this recurrence.
                Show your work.
\end{problem}

\begin{solution}

To set up the recurrence relation, let's look at how strings can end.


\end{solution}
%%%%%%%%%%%%%%%%%%%%%%%%%%%%
