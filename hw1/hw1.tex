
\documentclass{article}

\usepackage{fullpage,latexsym,picinpar,amsmath,amsfonts,amssymb}

\input{macros.tex}

\begin{document}

\centerline{\large \bf CS/MATH111 ASSIGNMENT 1}
\centerline{due Tuesday, January 20(11AM)}

\vskip 0.1in
\noindent{\bf Individual assignment:} Problems 1 and 2.

\noindent{\bf Group assignment:} Problems 1,2 and 3.

\vskip 0.2in

%%%%%%%%%%%%%%%%%%%%%%%%%%%%

\begin{problem}
(a)
Give the exact formula (as a function of $n$) for the number of
times ``bingo" is printed by Algorithm~\textsc{BingoPrint} below.
First express it as a summation formula and justify it. Then simplify it to 
obtain a closed-form expression. Show your derivation.

\noindent
(b)
Give the asymptotic value of the
number of ``bingo"s using the $\Theta$-notation. Include a brief justification. You will need a formula
for the sum of consecutive squares that you can find on the internet.

\begin{tabbing}
aa \= aa \= aa \= aa \= aa \= aa \= \kill
\textbf{Algorithm} \textsc{BingoPrint} $(n: \mbox{\bf integer})$ \\
      \> \textbf{for} $i \leftarrow 1$ \textbf{to} $2n+1$
                         \textbf{do} \\
      \> \> \textbf{for} $j \leftarrow 1$ \textbf{to} $i^2+2i$ \textbf{do} \\
      \> \> \> print(``bingo")
\end{tabbing}
\end{problem}

%%%%%%%%%%%%%%%%%%%%%%%%%%%%

\begin{problem}
Use mathematical induction to prove that $3^n \ge n2^n$ for $n\ge 0$. 
(Note: dealing with the base case may require some thought.)
\end{problem}

\begin{flushleft}
Base case:

First let's check P(2)/n=2.

$3^2 \ge 2\cdot2^2$
$9 \ge 8$

Which is always true.

\vskip 0.1in
Inductive step:

Let's assume P(k) / n=k (where k is also >=2) and is true.

$3^k \ge k2^k$

Then P(k+1) / n=k+1

$3^{k+1} \ge (k+1)2^{k+1}$ \\
$3\cdot3^k \ge ...$  \\
$3\cdot k 2^k \ge ...$ rewriting LHS in terms of assumption \\
$3k \ge (k+1)2$ divided $2^k$ \\
$k \ge 2$ distribute and subtract $2k$ \\
Which is always true.

\vskip 0.1in
Last cases:
We have n=0 and n=1 left to check in order to satisfy all solutions n>=0.
Let n=0.
$3^0 \ge 0\cdot2^0$
$1 >= 0$
Which is always true.
Let n=1.
$3^1 \ge 1\cdot2^1$
$3 \ge 2$
Which is always true.

\vskip 0.1in
$\therefore$ P holds in all cases.

\end{flushleft}



%%%%%%%%%%%%%%%%%%%%%%%%%%%%

\begin{problem}
Give the asymptotic values of the
following functions, using the $\Theta$-notation:
%
\begin{description}
%
\item{(a)} $9n^2 + n^3/2 + 29n + 13$
\item{(b)} $\sqrt{n}+ 7\log^5 n + 2n\log n$
\item{(c)} $1+ n^3\log^3n + 21 n^2\log^4n$
\item{(d)} $\log^7n + n 2^n + 13n\log^9n$
\item{(e)} $n^53^n+4^n$
%
\end{description}
%
Justify your answer.
Give an informal explanation using asymptotic
relations between the functions $n^c$, $\log n$, and $c^n$.
\end{problem}

%%%%%%%%%%%%%%%%%%%%%%%%%%%%

\vskip 0.1in
\paragraph{Submission.}
To submit the homework, you need to upload the pdf file into ilearn by 11AM on Tuesday, January 20,
\textbf{and} turn-in a paper copy in class.

\paragraph{Reminders.}
Remember that only {\LaTeX} papers are accepted. 
\end{document}

