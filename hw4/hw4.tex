\documentclass[11pt]{article}

\usepackage{fullpage,graphicx,latexsym,picinpar,amsbsy,amsmath,amsfonts}

           

%%%%%%%%%%%%%%%%%%%%%%%%%%%%%%%%%%%%%%%%%%%%%%%%%%%%%%%%%%%%%%%%%%%%%%%%%%%%%%%%%%%
%%%%%%%%%%%  LETTERS 
%%%%%%%%%%%%%%%%%%%%%%%%%%%%%%%%%%%%%%%%%%%%%%%%%%%%%%%%%%%%%%%%%%%%%%%%%%%%%%%%%%%

\newcommand{\barx}{{\bar x}}
\newcommand{\bary}{{\bar y}}
\newcommand{\barz}{{\bar z}}
\newcommand{\bart}{{\bar t}}

\newcommand{\bfP}{{\bf{P}}}

%%%%%%%%%%%%%%%%%%%%%%%%%%%%%%%%%%%%%%%%%%%%%%%%%%%%%%%%%%%%%%%%%%%%%%%%%%%%%%%%%%%
%%%%%%%%%%%%%%%%%%%%%%%%%%%%%%%%%%%%%%%%%%%%%%%%%%%%%%%%%%%%%%%%%%%%%%%%%%%%%%%%%%%
                                                                                
\newcommand{\parend}[1]{{\left( #1  \right) }}
\newcommand{\spparend}[1]{{\left(\, #1  \,\right) }}
\newcommand{\angled}[1]{{\left\langle #1  \right\rangle }}
\newcommand{\brackd}[1]{{\left[ #1  \right] }}
\newcommand{\spbrackd}[1]{{\left[\, #1  \,\right] }}
\newcommand{\braced}[1]{{\left\{ #1  \right\} }}
\newcommand{\leftbraced}[1]{{\left\{ #1  \right. }}
\newcommand{\floor}[1]{{\left\lfloor #1\right\rfloor}}
\newcommand{\ceiling}[1]{{\left\lceil #1\right\rceil}}
\newcommand{\barred}[1]{{\left|#1\right|}}
\newcommand{\doublebarred}[1]{{\left|\left|#1\right|\right|}}
\newcommand{\spaced}[1]{{\, #1\, }}
\newcommand{\suchthat}{{\spaced{|}}}
\newcommand{\numof}{{\sharp}}
\newcommand{\assign}{{\,\leftarrow\,}}
\newcommand{\myaccept}{{\mbox{\tiny accept}}}
\newcommand{\myreject}{{\mbox{\tiny reject}}}
\newcommand{\blanksymbol}{{\sqcup}}
                                                                                                                         
\newcommand{\veps}{{\varepsilon}}
\newcommand{\Sigmastar}{{\Sigma^\ast}}
                           
\newcommand{\half}{\mbox{$\frac{1}{2}$}}    
\newcommand{\threehalfs}{\mbox{$\frac{3}{2}$}}   
\newcommand{\domino}[2]{\left[\frac{#1}{#2}\right]}  

<<<<<<< HEAD
%%%%%%%%%%%% complexity classes

\newcommand{\PP}{\mathbb{P}}
\newcommand{\NP}{\mathbb{NP}}
\newcommand{\PSPACE}{\mathbb{PSPACE}}
\newcommand{\coNP}{\textrm{co}\mathbb{NP}}
\newcommand{\DLOG}{\mathbb{L}}
\newcommand{\NLOG}{\mathbb{NL}}
\newcommand{\NL}{\mathbb{NL}}

%%%%%%%%%%% decision problems

\newcommand{\PCP}{\sc{PCP}}
\newcommand{\Path}{\sc{Path}}
\newcommand{\GenGeo}{\sc{Generalized Geography}}

\newcommand{\malytm}{{\mbox{\tiny TM}}}
\newcommand{\malycfg}{{\mbox{\tiny CFG}}}
\newcommand{\Atm}{\mbox{\rm A}_\malytm}
\newcommand{\complAtm}{{\overline{\mbox{\rm A}}}_\malytm}
\newcommand{\AllCFG}{{\mbox{\sc All}}_\malycfg}
\newcommand{\complAllCFG}{{\overline{\mbox{\sc All}}}_\malycfg}
\newcommand{\complL}{{\bar L}}
\newcommand{\TQBF}{\mbox{\sc TQBF}}
\newcommand{\SAT}{\mbox{\sc SAT}}
=======
\newcommand{\naturals}{{\mathbb{N}}}
>>>>>>> 55ba746ad151af0d466421a34bb8baf6b90d7c8c

%%%%%%%%%%%%%%%%%%%%%%%%%%%%%%%%%%%%%%%%%%%%%%%%%%%%%%%%%%%%%%%%%%%%%%%%%%%%%%%%%%%
%%%%%%%%%%%%%%% for homeworks
%%%%%%%%%%%%%%%%%%%%%%%%%%%%%%%%%%%%%%%%%%%%%%%%%%%%%%%%%%%%%%%%%%%%%%%%%%%%%%%%%%%

\newcommand{\student}[2]{%
{\noindent\Large{ \emph{#1} SID {#2} } \hfill} \vskip 0.1in}

\newcommand{\assignment}[1]{\medskip\centerline{\large\bf CS 111 ASSIGNMENT {#1}}}

\newcommand{\duedate}[1]{{\centerline{due {#1}\medskip}}}     

\newcounter{problemnumber}                                                                                 

\newenvironment{problem}{{\vskip 0.1in \noindent
              \bf Problem~\addtocounter{problemnumber}{1}\arabic{problemnumber}:}}{}

\newcounter{solutionnumber}

\newenvironment{solution}{{\vskip 0.1in \noindent
             \bf Solution~\addtocounter{solutionnumber}{1}\arabic{solutionnumber}:}}
				{\ \newline\smallskip\lineacross\smallskip}

\newcommand{\lineacross}{\noindent\mbox{}\hrulefill\mbox{}}

\newcommand{\decproblem}[3]{%
\medskip
\noindent
\begin{list}{\hfill}{\setlength{\labelsep}{0in}
                       \setlength{\topsep}{0in}
                       \setlength{\partopsep}{0in}
                       \setlength{\leftmargin}{0in}
                       \setlength{\listparindent}{0in}
                       \setlength{\labelwidth}{0.5in}
                       \setlength{\itemindent}{0in}
                       \setlength{\itemsep}{0in}
                     }
\item{{{\sc{#1}}:}}
                \begin{list}{\hfill}{\setlength{\labelsep}{0.1in}
                       \setlength{\topsep}{0in}
                       \setlength{\partopsep}{0in}
                       \setlength{\leftmargin}{0.5in}
                       \setlength{\labelwidth}{0.5in}
                       \setlength{\listparindent}{0in}
                       \setlength{\itemindent}{0in}
                       \setlength{\itemsep}{0in}
                       }
                \item{{\em Instance:\ }}{#2}
                \item{{\em Query:\ }}{#3}
                \end{list}
\end{list}
\medskip
}

%%%%%%%%%%%%%%%%%%%%%%%%%%%%%%%%%%%%%%%%%%%%%%%%%%%%%%%%%%%%%%%%%%%%%%%%%%%%%%%%%%%
%%%%%%%%%%%%% for quizzes
%%%%%%%%%%%%%%%%%%%%%%%%%%%%%%%%%%%%%%%%%%%%%%%%%%%%%%%%%%%%%%%%%%%%%%%%%%%%%%%%%%%

\newcommand{\quizheader}{ {\large NAME: \hskip 3in SID:\hfill}
                                \newline\lineacross \medskip }


%%%%%%%%%%%%%%%%%%%%%%%%%%%%%%%%%%%%%%%%%%%%%%%%%%%%%%%%%%%%%%%%%%%%%%%%%%%%%%%%%%%
%%%%%%%%%%%%% for final
%%%%%%%%%%%%%%%%%%%%%%%%%%%%%%%%%%%%%%%%%%%%%%%%%%%%%%%%%%%%%%%%%%%%%%%%%%%%%%%%%%%

\newcommand{\namespace}{\noindent{\Large NAME: \hfill SID:\hskip 1.5in\ }\\\medskip\noindent\mbox{}\hrulefill\mbox{}}

<<<<<<< HEAD
=======

%%%%%%%%%%%%%%%%%%%%%%%%%%%%%%%%%%%%%%%%%%%%%%%%%%%%%%%%%%%%%%%%%%%%%%%%%%%%%%%%%%%
%%%%%%%%%%%%% for notes
%%%%%%%%%%%%%%%%%%%%%%%%%%%%%%%%%%%%%%%%%%%%%%%%%%%%%%%%%%%%%%%%%%%%%%%%%%%%%%%%%%%


\newtheorem{theorem}{Theorem}[section]
\newtheorem{definition}[theorem]{Definition}
\newtheorem{corollary}[theorem]{Corollary}
\newtheorem{lemma}[theorem]{Lemma}
\newtheorem{fact}[theorem]{Fact}
\newtheorem{claim}[theorem]{Claim}

\newenvironment{proof}{{\it Proof:\/}}{$\Box$\vskip 0.1in}

>>>>>>> 55ba746ad151af0d466421a34bb8baf6b90d7c8c


\begin{document}

% v -- YOUR NAME and SID in the braces
\student{ Chetas Manjunath}{ 860978339 }
% v -- YOUR NAME and SID in the braces
\student{ Craig Collier}{ 861100234 }   
% v -- ASSIGNMENT NUMBER in the braces
\assignment{ Assignment 4 } 
% v-- DUE DATE in the braces
\duedate{ 3/7/2015 }  

\medskip


\newcommand*{\Comb}[2]{{}^{#1}C_{#2}}%
%%%%%%%%%%%%%%%%%%%%%%%%%%%%%%%%%%%%%%%%%%%%%%%%%%%%%%%%%%%%%%%%%%%%%%%%%%


\begin{problem}
Give the asymptotic value (using the $\Theta$-notation)
for the number of letters that will be printed by the algorithms below.
Your solution needs to consist of an appropriate recurrence 
equation and its solution, with a brief justification.

\bigskip
\noindent
(a)\ \ 
\begin{minipage}[t]{3in}
\begin{tabbing}
aaa \= aaa \= aaa \= aaa \=  \kill
\textbf{Algorithm} \textsc{PrintXs} $(n: \mbox{\bf integer})$ \\
          \> \textbf{if} $n < 3$ \\
          \>\>  print(``X") \\
          \>\textbf{else} \\
          \>\>  \textsc{PrintXs}$(\ceiling{n/3})$\\
          \>\>  \textsc{PrintXs}$(\ceiling{n/3})$\\
          \>\>  \textsc{PrintXs}$(\ceiling{n/3})$\\
          \>\>  \textsc{PrintXs}$(\ceiling{n/3})$\\
      \>\> \textbf{for} $i \leftarrow 1$ \textbf{to} $2n$ \textbf{do} print(``X")
\end{tabbing}
\end{minipage}

\begin{solution}
\[ T(n) = aT( \frac{n}{b} ) + cn^{d} \]
\[ a = 4; b = 3; \]
\[ cn^{d} = 2n; d = 1; c = 2 \]
\[ L(n) = 4L( \frac{n}{3} ) + 2n \]
\[ a > b^{d} \]
\[ \theta(n^{log_{3}4}) \]
\end{solution}

\bigskip
\noindent
(b)\ \
\begin{minipage}[t]{3in}
\begin{tabbing}
aaa \= aaa \= aaa \= aaa \=  \kill
\textbf{Algorithm} \textsc{PrintYs} $(n: \mbox{\bf integer})$ \\
          \> \textbf{if} $n < 2$ \\
          \>\>  print(``Y") \\
          \>\textbf{else} \\
          \>\>  \textbf{for} $j \leftarrow 1$ \textbf{to} $7$ 
					\textbf{do} \textsc{PrintYs}$(\floor{n/2})$\\
      \>\> \textbf{for} $i \leftarrow 1$ \textbf{to} $n^3$ \textbf{do} print(``Y")
\end{tabbing}
\end{minipage}

\begin{solution}
\[ T(n) = aT( \frac{n}{b} ) + cn^{d} \]
\[ a = 7; b = 2; \]
\[ cn^{d} = n^{3}; d = 3; c = 1 \]
\[ L(n) = 7L( \frac{n}{2} ) + n^{3} \]
\[ a < b^{d} \]
\[ \theta(n^{3}) \]
\end{solution}


\bigskip
\noindent
(c)\ \ 
\begin{minipage}[t]{3in}
\begin{tabbing}
aaa \= aaa \= aaa \= aaa \=  \kill
\textbf{Algorithm} \textsc{PrintZs} $(n: \mbox{\bf integer})$ \\
          \> \textbf{if} $n < 2$ \\
          \>\>  print(``Z") \\
          \>\textbf{else} \\
          \>\>  \textbf{for} $j \leftarrow 1$ \textbf{to} $8$ 
					\textbf{do} \textsc{PrintZs}$(\floor{n/2})$\\
      \>\> \textbf{for} $i \leftarrow 1$ \textbf{to} $n^3$ \textbf{do} print(``Z")
\end{tabbing}
\end{minipage}

\begin{solution}
\[ T(n) = aT( \frac{n}{b} ) + cn^{d} \]
\[ a = 8; b = 2; \]
\[ cn^{d} = n^{3}; d = 3; c = 1 \]
\[ L(n) = 8L( \frac{n}{2} ) + n^{3} \]
\[ a = b^{d} \]
\[ \theta(n^{3} log n ) \]
\end{solution}

\bigskip
\noindent
(d)\ \ 
\begin{minipage}[t]{3in}
\begin{tabbing}
aaa \= aaa \= aaa \= aaa \=  \kill
\textbf{Algorithm} \textsc{PrintUs} $(n: \mbox{\bf integer})$ \\
          \> \textbf{if} $n < 4$ \\
          \>\>  print(``U") \\
          \>\textbf{else} \\
          \>\>  \textsc{PrintUs}$(\ceiling{n/4})$\\
          \>\>  \textsc{PrintUs}$(\floor{n/4})$\\
      \>\> \textbf{for} $i \leftarrow 1$ \textbf{to} $11$ \textbf{do} print(``U")
\end{tabbing}
\end{minipage}

\begin{solution}
\[ T(n) = aT( \frac{n}{b} ) + cn^{d} \]
\[ a = 2; b = 4; \]
\[ cn^{d} = 11; d = 0; c = 11 \]
\[ L(n) = 2L( \frac{n}{4} ) + 11 \]
\[ a > b^{d} \]
\[ \theta(n^{log_{4}2}) \]
\end{solution}

\bigskip
\ 


\bigskip
\noindent
(e)\ \ 
\begin{minipage}[t]{3in}
\begin{tabbing}
aaa \= aaa \= aaa \= aaa \=  \kill
\textbf{Algorithm} \textsc{PrintVs} $(n: \mbox{\bf integer})$ \\
          \> \textbf{if} $n < 3$ \\
          \>\>  print(``V") \\
          \>\textbf{else} \\
          \>\>  \textbf{for} $j \leftarrow 1$ \textbf{to} $10$ 
					\textbf{do} \textsc{PrintVs}$(\floor{n/3})$\\
      \>\> \textbf{for} $i \leftarrow 1$ \textbf{to} $2n^3$ \textbf{do} print(``V")
\end{tabbing}
\end{minipage}

\begin{solution}
\[ T(n) = aT( \frac{n}{b} ) + cn^{d} \]
\[ a = 10; b = 3; \]
\[ cn^{d} = 2n^{3}; d = 3; c = 2 \]
\[ L(n) = 10L( \frac{n}{3} ) + 2n^{3} \]
\[ a < b^{d} \]
\[ \theta(n^{3}) \]
\end{solution}


\bigskip
\


\end{problem}


%%%%%%%%%%%%%%%%%%%%%%%%%%%%

\begin{problem}
Determine (using the inclusion-exclusion principle)
the number of integer solutions of the equation:
%
\begin{eqnarray*}
x + y + z &=& 20,
\end{eqnarray*}
%
under the constraints 
%
\begin{eqnarray*}
        1 \;\le\; x \;\le\; 5 \\
        3 \;\le\; y \;\le\; 9 \\
        1\;\le\; z  \;\le\; 8
\end{eqnarray*}
%
Show your work.
\end{problem}

\begin{solution}

Re-writing the constraints, we get:

\[ 0 \leq x' \leq 4 \]
\[ 0 \leq y' \leq 6 \]
\[ 0 \leq z' \leq 7 \]

Substituting these constraints into the original sum, we get:

\[ (x' + 1) + (y' + 3) + (z' + 1) = 20 \]
\[ x' + y' + z' = 15 \]

The inclusion-exclusion principle tells us that:

\[ S(x' \leq 4 \wedge y' \leq 6 \wedge z' \leq 7) \]

Or

\[ S - S(x' \geq 5 \vee y' \geq 7 \vee z' \geq 8) \]
\[ = S - (S(x' \geq 5 ) + S(y' \geq 7) + S(z' \geq 8) - S(x' \geq 5 \wedge y' \geq 7) - S(x' \geq 5 \wedge z' \geq 8) - S(y' \geq 7 \wedge z' \geq 8) + S(x \geq 5 \wedge y' \geq 7 \wedge z' \geq 8)) \]

$S$ is equal to  $\binom{m+k-1}{k-1}$.

\[ S = \binom {15 + 3 -1}{3 - 1} = \binom{17}{2} = 136 \]

For each induvidual inequality, we use $\binom{m - A + k - 1}{k - 1}$

\[ = \binom{17}{2} - (\binom{12}{2} + \binom{10}{2} + \binom{9}{2} - \binom{5}{2} - \binom{4}{2} - \binom{2}{2} + \binom{0}{2}) \]
\[ = 136 - (66 + 45 + 36 - 10 - 6 - 1 + 0) \]
\[ = 136 - 130 \]
\[ = 6 \]
\end{solution}
%%%%%%%%%%%%%%%%%%%%%%%%%%%%

\begin{problem}
Determine (using the inclusion-exclusion principle)
the number of integer solutions of the equation:
%
\begin{eqnarray*}
x + y + z &=& 20,
\end{eqnarray*}
%
under the constraints 
%
\begin{eqnarray*}
        1 \;\le\; x \;\le\; 5 \\
        3 \;\le\; y \;\le\; 9 \\
        1\;\le\; z  \;\le\; 8
\end{eqnarray*}
%
Show your work.
\end{problem}

%%%%%%%%%%%%%%%%%%%%%%%%%%%%
\vskip 0.1in
\paragraph{Submission.}
To submit the homework, you need to upload the pdf file into ilearn by 8AM on Thursday, March 7,
and turn-in a paper copy in class.

\vfill

\end{document}
