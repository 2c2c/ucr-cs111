\documentclass[11pt]{article}

\usepackage{fullpage,graphicx,latexsym,picinpar,amsbsy,amsmath,amsfonts}

\input{macros.tex}

\begin{document}

% v -- YOUR NAME and SID in the braces
\student{ Chetas Manjunath}{ 860978339 }
% v -- YOUR NAME and SID in the braces
\student{ Craig Collier}{ 861100234 }   
% v -- ASSIGNMENT NUMBER in the braces
\assignment{ Assignment 4 } 
% v-- DUE DATE in the braces
\duedate{ 3/7/2015 }  

\medskip


\newcommand*{\Comb}[2]{{}^{#1}C_{#2}}%
%%%%%%%%%%%%%%%%%%%%%%%%%%%%%%%%%%%%%%%%%%%%%%%%%%%%%%%%%%%%%%%%%%%%%%%%%%


\begin{problem}
Give the asymptotic value (using the $\Theta$-notation)
for the number of letters that will be printed by the algorithms below.
Your solution needs to consist of an appropriate recurrence 
equation and its solution, with a brief justification.

\bigskip
\noindent
(a)\ \ 
\begin{minipage}[t]{3in}
\begin{tabbing}
aaa \= aaa \= aaa \= aaa \=  \kill
\textbf{Algorithm} \textsc{PrintXs} $(n: \mbox{\bf integer})$ \\
          \> \textbf{if} $n < 3$ \\
          \>\>  print(``X") \\
          \>\textbf{else} \\
          \>\>  \textsc{PrintXs}$(\ceiling{n/3})$\\
          \>\>  \textsc{PrintXs}$(\ceiling{n/3})$\\
          \>\>  \textsc{PrintXs}$(\ceiling{n/3})$\\
          \>\>  \textsc{PrintXs}$(\ceiling{n/3})$\\
      \>\> \textbf{for} $i \leftarrow 1$ \textbf{to} $2n$ \textbf{do} print(``X")
\end{tabbing}
\end{minipage}

\begin{solution}
\[ T(n) = aT( \frac{n}{b} ) + cn^{d} \]
\[ a = 4; b = 3; \]
\[ cn^{d} = 2n; d = 1; c = 2 \]
\[ L(n) = 4L( \frac{n}{3} ) + 2n \]
\[ a > b^{d} \]
\[ \theta(n^{log_{3}4}) \]
\end{solution}

\bigskip
\noindent
(b)\ \
\begin{minipage}[t]{3in}
\begin{tabbing}
aaa \= aaa \= aaa \= aaa \=  \kill
\textbf{Algorithm} \textsc{PrintYs} $(n: \mbox{\bf integer})$ \\
          \> \textbf{if} $n < 2$ \\
          \>\>  print(``Y") \\
          \>\textbf{else} \\
          \>\>  \textbf{for} $j \leftarrow 1$ \textbf{to} $7$ 
					\textbf{do} \textsc{PrintYs}$(\floor{n/2})$\\
      \>\> \textbf{for} $i \leftarrow 1$ \textbf{to} $n^3$ \textbf{do} print(``Y")
\end{tabbing}
\end{minipage}

\begin{solution}
\[ T(n) = aT( \frac{n}{b} ) + cn^{d} \]
\[ a = 7; b = 2; \]
\[ cn^{d} = n^{3}; d = 3; c = 1 \]
\[ L(n) = 7L( \frac{n}{2} ) + n^{3} \]
\[ a < b^{d} \]
\[ \theta(n^{3}) \]
\end{solution}


\bigskip
\noindent
(c)\ \ 
\begin{minipage}[t]{3in}
\begin{tabbing}
aaa \= aaa \= aaa \= aaa \=  \kill
\textbf{Algorithm} \textsc{PrintZs} $(n: \mbox{\bf integer})$ \\
          \> \textbf{if} $n < 2$ \\
          \>\>  print(``Z") \\
          \>\textbf{else} \\
          \>\>  \textbf{for} $j \leftarrow 1$ \textbf{to} $8$ 
					\textbf{do} \textsc{PrintZs}$(\floor{n/2})$\\
      \>\> \textbf{for} $i \leftarrow 1$ \textbf{to} $n^3$ \textbf{do} print(``Z")
\end{tabbing}
\end{minipage}

\begin{solution}
\[ T(n) = aT( \frac{n}{b} ) + cn^{d} \]
\[ a = 8; b = 2; \]
\[ cn^{d} = n^{3}; d = 3; c = 1 \]
\[ L(n) = 8L( \frac{n}{2} ) + n^{3} \]
\[ a = b^{d} \]
\[ \theta(n^{3} log n ) \]
\end{solution}

\bigskip
\noindent
(d)\ \ 
\begin{minipage}[t]{3in}
\begin{tabbing}
aaa \= aaa \= aaa \= aaa \=  \kill
\textbf{Algorithm} \textsc{PrintUs} $(n: \mbox{\bf integer})$ \\
          \> \textbf{if} $n < 4$ \\
          \>\>  print(``U") \\
          \>\textbf{else} \\
          \>\>  \textsc{PrintUs}$(\ceiling{n/4})$\\
          \>\>  \textsc{PrintUs}$(\floor{n/4})$\\
      \>\> \textbf{for} $i \leftarrow 1$ \textbf{to} $11$ \textbf{do} print(``U")
\end{tabbing}
\end{minipage}

\begin{solution}
\[ T(n) = aT( \frac{n}{b} ) + cn^{d} \]
\[ a = 2; b = 4; \]
\[ cn^{d} = 11; d = 0; c = 11 \]
\[ L(n) = 2L( \frac{n}{4} ) + 11 \]
\[ a > b^{d} \]
\[ \theta(n^{log_{4}2}) \]
\end{solution}

\bigskip
\ 


\bigskip
\noindent
(e)\ \ 
\begin{minipage}[t]{3in}
\begin{tabbing}
aaa \= aaa \= aaa \= aaa \=  \kill
\textbf{Algorithm} \textsc{PrintVs} $(n: \mbox{\bf integer})$ \\
          \> \textbf{if} $n < 3$ \\
          \>\>  print(``V") \\
          \>\textbf{else} \\
          \>\>  \textbf{for} $j \leftarrow 1$ \textbf{to} $10$ 
					\textbf{do} \textsc{PrintVs}$(\floor{n/3})$\\
      \>\> \textbf{for} $i \leftarrow 1$ \textbf{to} $2n^3$ \textbf{do} print(``V")
\end{tabbing}
\end{minipage}

\begin{solution}
\[ T(n) = aT( \frac{n}{b} ) + cn^{d} \]
\[ a = 10; b = 3; \]
\[ cn^{d} = 2n^{3}; d = 3; c = 2 \]
\[ L(n) = 10L( \frac{n}{3} ) + 2n^{3} \]
\[ a < b^{d} \]
\[ \theta(n^{3}) \]
\end{solution}


\bigskip
\


\end{problem}


%%%%%%%%%%%%%%%%%%%%%%%%%%%%

\begin{problem}
Determine (using the inclusion-exclusion principle)
the number of integer solutions of the equation:
%
\begin{eqnarray*}
x + y + z &=& 20,
\end{eqnarray*}
%
under the constraints 
%
\begin{eqnarray*}
        1 \;\le\; x \;\le\; 5 \\
        3 \;\le\; y \;\le\; 9 \\
        1\;\le\; z  \;\le\; 8
\end{eqnarray*}
%
Show your work.
\end{problem}

\begin{solution}

Re-writing the constraints, we get:

\[ 0 \leq x' \leq 4 \]
\[ 0 \leq y' \leq 6 \]
\[ 0 \leq z' \leq 7 \]

Substituting these constraints into the original sum, we get:

\[ (x' + 1) + (y' + 3) + (z' + 1) = 20 \]
\[ x' + y' + z' = 15 \]

The inclusion-exclusion principle tells us that:

\[ S(x' \leq 4 \wedge y' \leq 6 \wedge z' \leq 7) \]

Or

\[ S - S(x' \geq 5 \vee y' \geq 7 \vee z' \geq 8) \]
\[ = S - (S(x' \geq 5 ) + S(y' \geq 7) + S(z' \geq 8) - S(x' \geq 5 \wedge y' \geq 7) - S(x' \geq 5 \wedge z' \geq 8) - S(y' \geq 7 \wedge z' \geq 8) + S(x \geq 5 \wedge y' \geq 7 \wedge z' \geq 8)) \]

$S$ is equal to  $\binom{m+k-1}{k-1}$.

\[ S = \binom {15 + 3 -1}{3 - 1} = \binom{17}{2} = 136 \]

For each induvidual inequality, we use $\binom{m - A + k - 1}{k - 1}$

\[ = \binom{17}{2} - (\binom{12}{2} + \binom{10}{2} + \binom{9}{2} - \binom{5}{2} - \binom{4}{2} - \binom{2}{2} + \binom{0}{2}) \]
\[ = 136 - (66 + 45 + 36 - 10 - 6 - 1 + 0) \]
\[ = 136 - 130 \]
\[ = 6 \]
\end{solution}
%%%%%%%%%%%%%%%%%%%%%%%%%%%%

\begin{problem}
Determine (using the inclusion-exclusion principle)
the number of integer solutions of the equation:
%
\begin{eqnarray*}
x + y + z &=& 20,
\end{eqnarray*}
%
under the constraints 
%
\begin{eqnarray*}
        1 \;\le\; x \;\le\; 5 \\
        3 \;\le\; y \;\le\; 9 \\
        1\;\le\; z  \;\le\; 8
\end{eqnarray*}
%
Show your work.
\end{problem}

%%%%%%%%%%%%%%%%%%%%%%%%%%%%
\vskip 0.1in
\paragraph{Submission.}
To submit the homework, you need to upload the pdf file into ilearn by 8AM on Thursday, March 7,
and turn-in a paper copy in class.

\vfill

\end{document}
